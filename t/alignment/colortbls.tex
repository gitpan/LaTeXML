\documentclass{article}
\usepackage{colortbl}
\usepackage{dcolumn,longtable,hhline}
\begin{document}
\section{Basics}
\par\noindent
\begin{tabular}{|l|c|}
one&two\\
three&four\\
\end{tabular}

\par\noindent
\begin{tabular}{%
  |>{\columncolor[gray]{0.8}[0pt]}l%
  |>{\color{white}\columncolor[gray]{0.2}[0pt]}c|}
one&two\\
three&four\\
\end{tabular}

\section{Complex}

\par\noindent
\begin{tabular}{%
  |>{\columncolor[gray]{0.8}[.5\tabcolsep]}l%
  |>{\color{white}\columncolor[gray]{0.2}[.5\tabcolsep]}c|}
one&two\\
three&four\\
\end{tabular}

\par\noindent
\setlength\minrowclearance{2pt}
\newcolumntype{A}{%
  >{\color{white}\columncolor{red}[.5\tabcolsep]%
     \raggedright}%
  p{2cm}}
\newcolumntype{B}{%
  >{\columncolor{blue}[.5\tabcolsep]%
    \color{yellow}\raggedright}
  p{3cm}}
\newcolumntype{C}{%
  >{\columncolor{yellow}[.5\tabcolsep]}%
    D{.}{\cdot}{3.3}}
\newcolumntype{E}{%
  >{\large\bfseries%
    \columncolor{cyan}[.5\tabcolsep]}c}
\newcolumntype{F}{%
  >{\color{white}%
    \columncolor{magenta}[.5\tabcolsep]}c}
\newcolumntype{G}{%
  >{\columncolor[gray]{0.8}[.5\tabcolsep][\tabcolsep]}l}
\newcolumntype{H}{>{\columncolor[gray]{0.8}}l}
\newcolumntype{I}{%
  >{\columncolor[gray]{0.8}[\tabcolsep][.5\tabcolsep]}%
    D{.}{\cdot}{3.3}}

\begin{longtable}{ABC}
\multicolumn{3}{E}{A long table example}\\
\multicolumn{2}{F}{First two columns}&
\multicolumn{1}{F}{Third column}\\
\multicolumn{2}{F}{p-type}&
\multicolumn{1}{F}{D-type (\textsf{dcolumn})}\endfirsthead
\multicolumn{3}{E}{A long table example (continued)}\\
\multicolumn{2}{F}{First two columns}&
\multicolumn{1}{F}{Third column}\\
\multicolumn{2}{F}{p-type}&
\multicolumn{1}{F}{D-type (\textsf{dcolumn})}\endhead
\multicolumn{3}{E}{Continued\ldots}\endfoot
\multicolumn{3}{E}{The End}\endlastfoot
 P-column&and another one&12.34\\
\multicolumn{1}{G}{Total}&
\multicolumn{1}{H}{(wrong)}&
\multicolumn{1}{I}{100.6}\\
 Some long text in the first column&bbb&1.2\\
 aaa&and some long text in the second column&1.345\\
\multicolumn{1}{G}{Total}&
\multicolumn{1}{H}{(wrong)}&
\multicolumn{1}{I}{100.6}\\
 aaa&bbb&1.345\\
 Note that the coloured rules in all columns stretch to accomodate
large entries in one column. &bbb&1.345\\
 aaa&bbb&100\\
 aaa&Depending on your driver you may get unsightly gaps or lines
  where the  `screens' used to produce different shapes interact
  badly. You may want to cause adjacent panels of the same colour by
 specifying a larger overhang 
or by adding some negative space (in a "\string\noalign" between rows.&12.4\\
 aaa&bbb&45.3\\
\end{longtable}


\section{Row \& Cell Colors}
\par\noindent
\begin{tabular}{|l|c|}
one&two\\
\rowcolor{red}
three&four\\
\end{tabular}

\par\noindent
\begin{tabular}{%
  |l%
  |>{\columncolor{blue}}c|}
one&two\\
three&four\\
\end{tabular}

\par\noindent
\begin{tabular}{%
  |l%
  |>{\columncolor{blue}}c|}
one&two\\
\rowcolor{red}
three&four\\
\end{tabular}

\par\noindent
\begin{tabular}{%
  |l%
  |>{\columncolor{blue}}c|}
one&two\\
\rowcolor{red}
three&\cellcolor{green}four\\
\end{tabular}

\section{Gaudy}
\definecolor{orange}{cmyk}{0,0.61,0.87,0}
\definecolor{indigo}{cmyk}{0.8,0.9,0,0}
\definecolor{violet}{cmyk}{0.6,0.9,0,0}
\newcommand\rainbowline[1]{%
\hhline{%
>{\arrayrulecolor
{red}\doublerulesepcolor[rgb]{.3,.3,1}}%
|#1:=%
>{\arrayrulecolor{orange}\doublerulesepcolor[rgb]{.4,.4,1}}%
=%
>{\arrayrulecolor{yellow}\doublerulesepcolor[rgb]{.5,.5,1}}%
=%
>{\arrayrulecolor {green}\doublerulesepcolor[rgb]{.6,.6,1}}%
=%
>{\arrayrulecolor {blue}\doublerulesepcolor[rgb]{.7,.7,1}}%
=%
>{\arrayrulecolor{indigo}\doublerulesepcolor[rgb]{.8,.8,1}}%
=%
>{\arrayrulecolor{violet}\doublerulesepcolor[rgb]{.9,.9,1}}%
=:#1|%
}}
\arrayrulecolor{red}
\doublerulesepcolor[rgb]{.3,.3,1}%
\begin{tabular}{||*7{>{\columncolor[gray]{.9}}c}||}
\rainbowline{t}%
\arrayrulecolor{violet}\doublerulesepcolor[rgb]{.9,.9,1}
Richard&of&York&gave&battle&in&
\multicolumn{1}{>{\columncolor[gray]{.9}}c||}{vain}\\
\rainbowline{}%
1&2&3&4&5&6&
\multicolumn{1}{>{\columncolor[gray]{.9}}c||}{7}\\
\rainbowline{b}%
\end{tabular}

\end{document}
